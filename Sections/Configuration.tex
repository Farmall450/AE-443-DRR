\textcolor{red}{
\begin{itemize}
    \item Discuss external configuration alternatives considered and design choices made.
    \item Discuss selected aircraft configuration design (i.e. distinctions between variants, major features, design characteristics/objectives).
    \item Include at least one trade study that uses quantitative or qualitative analysis to support design decisions made.
    \item Discuss the landing gear philosophy.
    \item 3-View drawing (VSP acceptable).
    \item Discuss future work
\end{itemize}}

Many configurations of aircraft were considered during initial design phase. First, a trade study was performed over similar aircraft that are either currently in service or have flown in the past. Table \ref{tabmk1} describes seven aircraft and some of their external configuration characteristics. 

\begin{table}[H]
    \centering
    \caption{Quantitative Trade Study of Similar Aircraft}
    \begin{tabular}{|m{3cm}|c|c|c|c|c|}
    \hline
    \label{tabmk1}
    \textbf{Aircraft} & \textbf{Tail Type} & \textbf{Wing Type} & \textbf{\# of Engines} & \textbf{2-class Seating} & \textbf{\# of Decks} \\
    \hline
    Boeing 777-200 & Conventional & Low & 2 & 375 & Single \\
    \hline
    Airbus 340-600 & Conventional & Low & 4 & 440 & Single \\
    \hline
    Boeing 787-10 & Conventional & Low & 2 & 330 & Single \\
    \hline
    McDonnell Douglas MD-11 & Conventional (w/ engine) & Low & 3 & 323 & Single \\
    \hline
    Lockheed 1011-100 & Conventional (w/ engine) & Low & 3 & 304 & Single \\
    \hline
    Boeing 747-400 & Conventional & Low & 4 & 496 & Double \\
    \hline
    Airbus 350-900 & Conventional & Low & 2 & 315 & Single \\
    \hline
    \end{tabular}
\end{table}

From the trade study, all seven of described aircraft featured a low-level wing along with a conventional tail. This is most likely due to the low-level wing providing space for a retractable landing gear along with simplicity of design. Thus, the has decided to go with a low-level wing.

The next external configuration decision was between a T-tail conventional tail. While the T-tail does reduce the risk of the tail stalling, it is much more difficult to design as the vertical tail has to support the entire weight of the horizontal tail. Alternatively, conventional tails are lighter and provide more simplicity in terms of design. Another benefit with conventional tails is that it offers more space to store fuel for the aircraft. As shown from the trade study, conventional tails are very widely used in commercial aircraft in today's market. Thus, the team has decided to use a conventional tail configuration. 

Furthermore, the number of engines on our aircraft along with the number of decks on the aircraft were to be decided. The team decided to go with a single-decker to provide simplicity in design as well reduced cost in manufacturing. Moreover, two engines were chosen for our aircraft, mainly for the reduced maintenance cost of more engines as well as a smaller fuel cost. The proven reliability of modern engines in trans-oceanic flights also played a role in deciding to have two engines. Figure ....... shows an OpenVSP model of the aircraft. 
