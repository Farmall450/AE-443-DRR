\subsection{Comparison Analysis}
\label{subsection: comparison}
Using the aircraft in Table \ref{tab:trade_params}, the fidelity of the mass estimation equations found in Roskam Part 2 \cite{roskam_2}, Roskam Part 5 \cite{roskam_5}, and Raymer \cite{raymer} can be quantified. By choosing the equations which closely fit the values seen in the real aircraft from the trade study rather than equations which may over or under estimate the mass, a higher fidelity prediction of the total mass of the aircraft can be created. At this time, little analysis has been done in terms of mass properties of the aircraft.

\subsection{Future Work}
Future work includes several tasks, such as creating a parametric spreadsheet using equations found in Roskam and Raymer and key parameters from the aircraft size to estimate the center of gravity location and the weight of the aircraft. This method would provide a low fidelity estimation, but by using the best equation for each section of the aircraft as described in \ref{subsection: comparison}, a higher fidelity solution can be reached. Another method of estimating the mass properties of the aircraft would be to create a high-detailed CAD model of the aircraft, where materials are assigned to each part of the structure with enough detail such that the weight, center of gravity, and moments of inertia can be estimated via the CAD software. 






% \textcolor{red}{
% \begin{itemize}
%     \item Discuss any analysis supporting the sizing analysis.
%     \begin{itemize}
%         \item Mass property methods chosen for different parts of the aircraft.
%         \item Estimate CG location.
%     \end{itemize}}
%     \item Discuss future work.
%     \item AIAA: Aircraft weight statement, aircraft center-of-gravity envelope reflecting payloads and fuel allocation. Establish a forward and aft center of gravity (CG) limits for safe flight. (may come later...con't on AIAA doc)
% \end{itemize}}