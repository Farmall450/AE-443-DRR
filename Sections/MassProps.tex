\subsection{Comparison Analysis}
\label{subsection: comparison}
Given the preliminary geometry of the aircraft, the mass properties of the aircraft were calculated and tabulated in Table \ref{tab:mass_props}. In these calculations, the only source of equations used and analyzed came from Raymer \cite{raymer}, equations from Roskam (both Part II \cite{roskam_2} and Part V \cite{roskam_5}) are not taken into account. 

\begin{table}[!h]
\centering
\caption{Mass Estimations of SAM Mk1}
\begin{tabular}{|p{4cm}||p{2cm}|p{5cm}| }
\toprule
\multicolumn{1}{|c||}{\textbf{Description}} & \multicolumn{1}{c|}{\textbf{Weight (lbs)}} &  
\multicolumn{1}{c|}{\textbf{Equation Used}} \\ \hline \hline 
Wing & 40,000 & Raymer Table 15.2 \\ \hline
Fuselage & 61,600 & 25 lb/in (Raymer estimation) \\ \hline
Horizontal Stabilizer & 6,000 & 6 lb/sq ft (Raymer estimation) \\ \hline
Vertical Stabilizer & 2,500 & 5 lb/sq ft (Raymer estimation) \\ \hline
Main Landing Gear & 4,100 & Raymer 15.50 \\ \hline
Nose Landing Gear & 450 & Raymer 15.51 \\ \hline
Avionics & 1,330 & Raymer 15.57 \\ \hline
Electrical System & 510 & Raymer 15.56 \\ \hline
Anit-Icing and AC System & 4,700 & Raymer 15.58 \\ \hline
Engine Weight & 19,000 & Estimation (from sizing) \\ \hline
Baggage Weight & 12,300 & 30 lbs/person (per RFP) \\ \hline
Payload Weight & 82,000 & 200 lbs/person (per RFP) \\ \hline
\textbf{Empty Weight} & 148,100 &  \\ \hline
\textbf{MTOW} & 305,100 &  \\ \hline
\textbf{MZFW} & 230,100 &  \\ \hline
\textbf{MLW} & 200,000 &  \\
\bottomrule
\end{tabular}
\label{tab:mass_props}
\end{table}
\FloatBarrier

The CG of the aircraft was also estimated using the estimation provided by Raymer \cite{raymer}, which states the CG is roughly located at 30\% MAC from the tip of the root chord. From the computer model of the aircraft, the CG is estimated to be 1034 in (86 ft) from the nose of the aircraft, or 1174 in (98 ft) from the aircraft coordinate system.


\subsection{Future Work}
Using the aircraft in Table \ref{tab:trade_params} and other seed aircraft, the fidelity of the mass estimation equations found in Roskam Part 2 \cite{roskam_2}, Roskam Part 5 \cite{roskam_5}, and Raymer \cite{raymer} can be quantified. By choosing the equations which closely fit the values seen in the real aircraft from the trade study rather than equations which may over or under estimate the mass, a higher fidelity prediction of the total mass of the aircraft can be created. Another method of estimating the mass properties of the aircraft would be to create a high-detailed CAD model of the aircraft, where materials are assigned to each part of the structure with enough detail such that the weight, CG, and moments of inertia can be estimated via the CAD software. 






% \textcolor{red}{
% \begin{itemize}
%     \item Discuss any analysis supporting the sizing analysis.
%     \begin{itemize}
%         \item Mass property methods chosen for different parts of the aircraft.
%         \item Estimate CG location.
%     \end{itemize}}
%     \item Discuss future work.
%     \item AIAA: Aircraft weight statement, aircraft center-of-gravity envelope reflecting payloads and fuel allocation. Establish a forward and aft center of gravity (CG) limits for safe flight. (may come later...con't on AIAA doc)
% \end{itemize}}