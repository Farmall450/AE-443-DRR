\subsection{Configuration}
The aircraft will feature three sets of retractable landing gear customary of that found on similar aircraft within the aircraft featured in the trade study found in Table \ref{tab:trade_params}. Tentatively, the front assembly will be composed of two wheels, and the rears two symmetric sets of six wheels each, three wheels in a line on each side.  All of the gear will be hydraulically mounted to dampen the impact seen upon touchdown.  The "tricycle" arrangement is traditionally seen due to both the stability of three on uneven surfaces as well as the weight distribution of the aircraft.  The front gear will also feature the main taxi light.  

\subsection{Future Work}
Future work will consist of an analysis of the load and stress placed on the landing gear during taxi, takeoff, and, most importantly, landing.  Additional consideration will be taken to ensure the landing gear and its related hydraulic systems stow within the contour of the fuselage as developed by aerodynamics. The necessary kinematics of the gear will be studied. Furthermore, brake sizing and fitment inside the wheel will be verified. The number, size, and type of wheels and number of struts needed must be finalized. Lastly, the height of the landing gear will be determined by the takeoff and flare angle, the placement of the gear, and length of the fuselage.


% \textcolor{red}{
% \begin{itemize}
%     \item Discuss landing gear placement and design approach.\checkmark (\textit{JC})
%     \item Discuss future work.\checkmark (\textit{JC})
% \end{itemize}}