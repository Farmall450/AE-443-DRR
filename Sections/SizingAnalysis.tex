The initial sizing analysis was conducted by examining a variety of similar aircraft and making sure to satisfy all the requirements given by the AIAA RFP \cite{RFP}. Some of the analyzed aircraft included the Boeing 777-200, Boeing 787-100, Airbus A340-600, and the Lockheed L-1011-100, which all featured similar capacities and ranges.  Once compiled, the aircraft were assigned a quantitative rank from one to seven based on  capacity and range. Sorting by composite score, the Boeing 777-200 was chosen as the main seed aircraft due to its highest perceived similarity to a hypothetical plane developed for the RFP. Some key parameters derived from seed analysis and RFP mandates are presented in Table \ref{tab:req}; a list of all candidate aircraft considered in the sizing analysis are listed in Table \ref{tabmk1}, as the same subjects were used in the Section \ref{section: Configuration} trade study.

\begin{table}[h!] 
    \centering
    \caption{Key Parameters}
    \begin{tabular}{ |c||c| }\toprule
    \textbf{Parameter} & \textbf{Value} \\\hline\hline
    Payload & 94,000 lb  \\\hline
    Range & 3,500 nm \\\hline
    Takeoff Field Length with a 35 ft obstacle & 9,000 ft  \\\hline
    Landing Field Length & 9,000 ft \\\hline
    Cruise Speed & 487 kts\\\hline
    Cruise Altitude & 37,000 ft \\\hline
    Max Takeoff Weight & 550,000 lb\\\hline 
    L/D & 17.83\\\hline 
    $C_{L_{max}}$ & 1.65\\\hline 
    SFC & 0.65\\\hline 

    \end{tabular}\label{tab:req}
\end{table}

Parameters compiled from the similarity analysis were approximated by taking historic averages of those respective to the different analyzed aircraft. Preliminary cruise speed was obtained by assuming an average cruise Mach of 0.9, while initial cruise altitude was estimated by taking averages of similar aircraft. Many other sizing parameters, specifically the wing geometries, were calculated using equations from Raymer \cite{raymer}. The wing area, however, was calculated by taking a couple averages of seed aircraft as an initial value of 4,600 ft$^2$. This value was used to determine other wing related parameters, and was fine tuned as the sizing process went along. With all estimated parameters, an iterative process was used with a sizing spreadsheet to converge all sizing values for the initial Sam Mark I. This step was necessary due to the significant differences in range from all contemporary widebody, 300+ capacity commercial aircraft which could have been used as a seed.  The aircraft's weights were roughly calculated using a build up method, instead of taking the seed aircraft's weights, equations from Raymer \cite{raymer} and an initial assumption of 75,000 lb of fuel. A separate trade study was also conducted to address some key sizing parameters regarding the aircraft's fuselage to make sure that the Sam Mark I featured enough room for the designed seating arrangement. The trade study is demonstrated in Table \ref{tab:trade_params}.

\begin{table}[!h] 
    \centering
    \caption{Trade Study}
    \begin{tabular}{ |c||c||c| }\toprule
    \textbf{Aircraft} & \textbf{Fuselage Length [ft]} & \textbf{Max Takeoff Weight [lb]} \\\hline\hline
    Boeing 777-200 & 205.7 & 535,087  \\\hline
    Boeing 787-100 & 224.0 & 560,000  \\\hline
    Lockheed L-1011-100 & 177.8 & 466,079  \\\hline
    Airbus A340-600 & 228.2 & 390,307  \\\hline

    \end{tabular}\label{tab:trade_params}
\end{table}

\newpage

After analyzing multiple aircraft, a fuselage length of 210 ft was decided in order to hold more passengers than the Boeing 777-200, and to support a 3-4-3 economy seat arrangement. After utilizing a weight build up method, a max takeoff weight of 550,000 lb was calculated, which is about 15,000 lb heavier than the main seed aircraft. Once the weights were calculated, the total drag was able to be computed at desired takeoff conditions. This allowed for a computation of required takeoff thrust to be about 71,500 lbf. 


% \textcolor{red}{\begin{itemize}
%     \item AIAA: A description or graphical representation of the aircraft sizing based on the
% requirements and design objectives given. \hl{This should describe or represent the
% quantitative justification for the wing area and thrust of the aircraft that was selected.} 
%     \item Discuss the team’s similarity analysis and the rationale behind the aircraft selection. \checkmark JJ
%     \item Discuss what data was extracted from the similarity analysis. \checkmark JJ
%     \item Discuss a part by part build up and how it compares to the seed analysis \checkmark JJ
%     \item Discuss initial sizing and constraint analysis. 
%     \begin{itemize}
%         \item Include a table of key parameters used with justification for their values. \checkmark JJ
%         \item Include at least takeoff, landing, cruise, service ceiling, and \hl{loiter constraints.}
%     \end{itemize}
%     \item Include at least two trade studies that uses quantitative analysis to support design decisions made. \hl{I split into the sizing \& then fuse sizing as the "second" one}  Maybe tie into the trade study Mat outlined right below? I put a possible bridge in the first paragraph to it if you want to go that route. 
%     \item Include the aircraft allocated and derived requirements and their justifications: \hl{(I think he might want this for all of the planes we used? Open to interpretation)}
%     \begin{itemize}
%         \item $\frac{L}{D}$
%         \item $C_{L,max}$
%         \item SFC
%         \item Weights
%     \end{itemize}
% \end{itemize}}