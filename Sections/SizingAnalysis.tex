The initial sizing analysis was conducted by examining a variety of similar aircraft and making sure to satisfy all the requirements given by the RFP. Some of the analyzed aircraft included the Boeing 777-200, Boeing 787-100, Airbus A340-600, and the Lockheed L-1011-100, which all contained similar capacities and ranges. Ultimately, the Boeing 777-200 was chosen as the main seed aircraft due to its similar range to the RFP requirement, and its large cabin size. Some key parameters derived from seed analysis are presented in table \ref{tab:req}. 

\begin{table}[h!] 
    \centering
    \caption{Key Parameters}
    \begin{tabular}{ |c||c| }\toprule
    \textbf{Parameter} & \textbf{Value} \\\hline\hline
    Payload & 94,000 lb  \\\hline
    Range & 3,500 nm \\\hline
    Takeoff Field Length with a 35 ft obstacle & 9,000 ft  \\\hline
    Landing Field Length & 9,000 ft \\\hline
    Cruise Speed & 487 kts\\\hline
    Cruise Altitude & 37,000 ft \\\hline
    Max Takeoff Weight & 580,900 lb\\\hline 
    L/D & 17.83\\\hline 
    $C_{L_{max}}$ & 1.65\\\hline 
    SFC & 0.65\\\hline 

    \end{tabular}\label{tab:req}
\end{table}

Parameters that were not able to be compiled from the similarity analysis were approximated by taking historic averages of different analyzed aircraft. Parameters such as cruise speed, were estimated by assuming an average cruise Mach of .85, while cruise altitude was estimated by taking averages of similar aircraft. Many other sizing parameters, specifically the wing geometries, were calculated using equations from Raymer. With all estimated parameters, an iterative process was used with a sizing spreadsheet to converge all sizing values for the Sam Mark I. The aircraft's weights were calculated using a build up method. Instead of taking the seed aircraft's weights, equations from Raymer and Roskom were utilized to build up each aircraft part weight. A trade study was also conducted for some key sizing parameters to make sure that the Sam Mark I has enough room for the designed seat arrangement. The trade study is demonstrated in table \ref{tab:req}.

\begin{table}[h!] 
    \centering
    \caption{Trade Study}
    \begin{tabular}{ |c||c||c| }\toprule
    \textbf{Aircraft} & \textbf{Fuselage Length [ft]} & \textbf{Max Takeoff Weight [lb]} \\\hline\hline
    Boeing 777-200 & 205.7 & 535,087.3  \\\hline
    Boeing 787-100 & 224 & 560,000  \\\hline
    Lockheed L-1011-100 & 177.8 & 466,079.6  \\\hline
    Airbus A340-600 & 228.2 & 390,307  \\\hline

    \end{tabular}\label{tab:req}
\end{table}

After analyzing multiple aircraft, a fuselage length of 240 ft was decided in order to hold more passengers than the Boeing 777-200, and to support a 3-4-3 seat arrangement. After utilizing a weight build up method, a Max takeoff weight of 580,900 lb was calculated, which is about 50,000 lb heavier than the main seed aircraft. 

\textcolor{red}{\begin{itemize}
    \item AIAA: A description or graphical representation of the aircraft sizing based on the
requirements and design objectives given. This should describe or represent the
quantitative justification for the wing area and thrust of the aircraft that was selected.
    \item Discuss the team’s similarity analysis and the rationale behind the aircraft selection.
    \item Discuss what data was extracted from the similarity analysis.
    \item Discuss a part by part build up and how it compares to the seed analysis
    \item Discuss initial sizing and constraint analysis.
    \begin{itemize}
        \item Include a table of key parameters used with justification for their values.
        \item Include discussion of modifications made to account for the hybrid-electric nature of your aircraft.
        \item Include at least takeoff, landing, cruise, service ceiling, and loiter constraints.
    \end{itemize}
    \item Include at least two trade studies that uses quantitative analysis to support design decisions made.
    \item Include the aircraft allocated and derived requirements and their justifications:
    \begin{itemize}
        \item $\frac{L}{D}$
        \item $C_{L,max}$
        \item SFC
        \item Weights
    \end{itemize}
\end{itemize}}