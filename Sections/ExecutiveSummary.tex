The following design report represents the basis for a new aircraft developed by Team Toucan at the University of Illinois at Urbana-Champaign in response to AIAA's Request for Proposal\cite{RFP} addressing the industry need for a high capacity, short to medium haul widebody capable of mitigating the increased traffic and congestion found at many modern airports.  While the design aircraft will be capable of carrying up to 400 passengers up to 3,500 nm, it has been engineered to bridge the gap between low capacity, short range regional jets and high capacity, long range twin jets currently on the market.  This report validates the possibility of addressing the prescribed requirements in a commercial aircraft.

Prominent driving design points of the Sam Mark I include takeoff and landing distances under 9,000 ft, maximum range of 3,500 nm, capacity of 400 passengers and their respective luggage, 10 person crew (two pilots, eight flight attendants), starting cruise altitude of 37,000 ft, adherence to FAA 14 CFR Part 25 regulations and fundamental desire to maximize customer value through minimizing fuel consumption as well as other operating costs.  Cabin pressure will not exceed 8,000 ft, and five ft$^3$ of luggage space will be available per customer, increasing passenger comfort.  The proposed aircraft will be service ready by 2029 and has been designed as a twin engine, twin aisle widebody with a traditional tail.  Most notably, the entire wing including its sub-structure will be designed and manufactured utilizing modern composites to minimize structural weight while not sacrificing performance or longevity.  Other value-added design points include leading and trailing edge high lift devices, VFR and IFR flight with autopilot, and de-icing capability for flight in less than ideal weather conditions.

To date, preliminary sizing of the aircraft is nearing completion.  Important current parameters to define the aircraft include a cruise speed of Mach 0.9, MTOW weight of 550,000 lb, of which 75,000 lb are fuel, AR of nine, wing span of 212 ft, and mean aerodynamic chord of 328 in. The seed and fundamental based estimations in this report are fluid, and moving forward Team Toucan will be performing in-depth analysis of all major components to fully optimize the design to the driving design points.  Leading edge slats and trailing edge,  Fowler-type flaps will be designed and integrated into the wing structure to efficiently generate lift under critical conditions such as take-off and landing.  Final design will follow a detailed FEA and CFD analysis of the aircraft's aerodynamic surfaces and structure in order to ensure flight worthiness and will accompany a thorough approximation of assembly cost, timeline, and life cycle maintenance costs.  Additionally, a component wise weight build up of corrected values for hybrid construction will supplement an in-depth stability and control analysis.  Considerable effort will be put into meeting or exceeding applicable FAA 14 CFR Part 25 requirements for commercial aircraft, and final design will be completed within the next two months, with the vast majority complete in the next four weeks leaving adequate time for review, system integration check, and submission of a safe, value-driven aircraft design by May 14th, 2020.

\textcolor{red}{
\begin{itemize}
    \item Briefly discuss motivation for the aircraft, key requirements, and your design drivers. (JC \checkmark)
    \item Briefly  summarize the proposed aircraft, including key  characteristics (e.g. general configuration, \textbf{TOGW, primary dimensions), key performance metrics, key differentiators, and a cost summary.} (JC \checkmark)
    \item Include any requirements not met, struggling with, or have chosen to exceed. \textbf{(none?)}
    \item Discuss the near term milestones and provide a project timeline for the semester. 
    \item Maximum length: one page, single-spaced if necessary. (JC \checkmark)
\end{itemize}}