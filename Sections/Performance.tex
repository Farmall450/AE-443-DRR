\textcolor{red}{
\begin{itemize}
    \item Discuss performance analysis for takeoff and cruise supporting the sizing analysis.
    \item Discuss future work.
    \item AIAA: Aircraft performance summaries shall be documented and the aircraft flight envelope
    shall be shown graphically
    \item AIAA: Payload range chart(s)
    \item AIAA: Important aerodynamic characteristics and aerodynamic performance for key mission
    segments and requirements (shared w/ aero)
\end{itemize}}

The AIAA RFP \cite{RFP} requires this short range aircraft to have a range of at least 3,500 nm plus sufficient reserves while carrying a maximum of 400 passengers, a flight crew of 10, as well as their combined baggage.  Furthermore, both takeoff and landing distances are restrained to 9,000 ft or less off asphalt or concrete runways.  First-cut analysis was performed using an estimated empty weight as well as a fuel weight calculated by using a TSFC average of contemporary large turbofan engines, such as what would be found on an aircraft of this size.  This data was then input into an iterative spreadsheet to converge on an optimal design for a bounded range.  For this approximation, cruise altitude began at FL370 following by XYZ subsequent step climbs.  Climb, descent, and loiter follow the mission profile in <<section of conops>> per FAA requirements (? or however originally determined).  Elaborate expected performance

\textbf{Prior teams split this into two subsets: required perf (what I put, essentially) and expected perf (via spreadsheet).}