\documentclass[conf]{new-aiaa}
%\documentclass[journal]{new-aiaa} for journal papers
\usepackage[utf8]{inputenc}

\usepackage{graphicx}
\usepackage{amsmath}
\usepackage[version=4]{mhchem}
\usepackage{siunitx}
\usepackage{longtable,tabularx}
\usepackage{xcolor}
\setlength\LTleft{0pt} 

\begin{document}

\begin{titlepage}
    \begin{center}
        \vspace*{1cm}
        \large
        \textbf{TITLE} \\
        \vspace{0.5cm}
        \textbf{Team Name} \\
        \vspace{0.5cm}
        \normalsize
        \textbf{by} \\
        \vspace{0.5cm}
        \textbf{Names} \\
        \vspace{0.5cm}
        \textbf{Course Name}
        \vspace{0.5cm}
        \textbf{Date} \\
        \vspace{2.5cm}
        $\bullet$\\
        \vspace{2.5cm}
        $\bullet$\\
        \vspace{2.5cm}
        $\bullet$\\
        \vspace{1.75cm}
        \textbf{Include drawings or team logo}
    \end{center}
\end{titlepage}

\newpage

\pagenumbering{roman}
\section{Group Member Assignments Sheet}

\textcolor{red}{Include  a  list  of  team  members,  their  signatures,  and  primary  and  secondary  responsibilities.  Begin front matter pagination in Roman numerals on this page; i.e. i, ii, iii, iv, ...}
\newpage

\tableofcontents

\newpage

\section{Nomenclature}
\hspace{-0.5in}\textbf{Coefficients}
{\renewcommand\arraystretch{1.0}
\noindent\begin{longtable*}{@{}l @{\quad=\quad} l@{}}
    AR & Aspect Ratio \\
    % b & Wing Span \\
    % c & Chord length \\
    % $C_D$ & Wing Drag Coefficient \\
    % $C_d$ & Airfoil Drag Coefficient \\
    % $C_{D,i}$ & Induced Drag Coefficient \\
    % $C_f$ & Skin Friction Coefficient \\
    % $C_L$ & Wing Lift Coefficient \\
    % $C_l$ & Airfoil Lift Coefficient \\
    % e & Oswald Efficiency Factor \\
    % FF & Form Factor Term for parasitic drag calculation \\
    % g & Gravity (Earth = 9.81 $\frac{m}{s^2}$ \\
    % $h_p$ & Pressure Altitude \\
    % Isp & Specific Impulse \\
    % K & Drag due to lift factor \\
    % L/D & Lift to Drag ratio \\
    % LE & Leading Edge \\
    % M & Mach Number \\
    % m & Total aircraft mass \\
    % MAC & Mean aerodynamic chord \\
    % $M_{cr}$ & Critical Mach number \\
    % $M_{dd}$ & Drag Divergent Mach number \\
    % n & Load factor \\
    % Q & Interference factor for parasitic drag calculation \\
    % $q_{\infty}$ & Dynamic pressure \\
    % Re & Reynolds Number \\
    % S & LE Suction \\
    % $\mathcal{S}$ & Wing Area \\
    % SFC & Specific fuel consumption \\
    % T/W & Thrust-to-weight ratio \\
    % t & Airfoil thickness \\
    % t/c & Airfoil thickness-to-chord length ratio \\
    % TE & Trailing Edge \\
    % TO & Take-off \\
    % $V_{NE}$ & Never Exceed Speed \\
    % $V_{1}$ & TO Decision Speed \\
    % $V_2$ & TO Safety Speed \\
    % $W_e$ & Empty Weight\\
    % $W_f$ & Fuel weight \\
    % $W_o$ & TO Gross Weight \\
    % W/S & Wing Loading \\
\end{longtable*}}
\hspace{-0.5in}\textbf{Greek Symbols}
{\renewcommand\arraystretch{1.0}
\noindent\begin{longtable*}{@{}l @{\quad=\quad} l@{}}
    $\alpha$ & Angle of Attack \\
    % $\beta$ & Prandtl-Glauert Compressibility correction \\
    % $\delta_f$ & Flap Deflection \\
    % $\Lambda$ & Sweep Angle \\
    % $\epsilon$ & Unit Strain \\
    % $\gamma$ & Shearing Strain \\
    % $\Gamma$ & Wing Dihedral Angle \\
    % $\eta$ & Efficiency \\
    % $\lambda$ & Wing Taper Ratio ($C_{tip} / C_{root}$) \\
    % $\mu$ & Viscosity \\
    % $\rho$ & Air Density \\
    % $\sigma$ & Air density ratio ($=\rho/\rho_o$) \\
    % $\tau$ & Unit Shear Stress \\
\end{longtable*}}
\hspace{-0.5in}\textbf{Abbreviations and Acronyms}
{\renewcommand\arraystretch{1.0}
\noindent\begin{longtable*}{@{}l @{\quad=\quad} l@{}}
    A/C & Aircraft \\
    % AEI & All Engines Inoperative \\
    % APU & Auxiliary Power Unit \\
    % BFL & Balanced Field Length \\
    % BL & Boundary Layer \\
    % BPR & Turbofan engine bypass ratio \\
    % CAD & Computer Aided Design \\
    % CAS & Calibrated Airspeed \\
    % CER & Cost Estimated Relationship \\
    % CFD & Computational Fluid Dynamics \\
    % CG & Center of Gravity \\
    % ConOps & Concept of Operations \\
    % CTOL & Conventional Take-off and Landing \\
    % DARPA & Defense Advanced Research Projects Agency \\
    % DoD & Department of Defense \\
    % EAS & Equivalent Airspeed \\
    % FAA & Federal Aviation Administration \\
    % FAR & Federal Aviation Regulations (certification Spces) \\
    % FAR & Federal Acquisition Regulations \\
    % FBW & Fly By Wire \\
    % FOD & Foreign Object Damage \\
    % GA & General Aviation \\
    % GPS & Global Positioning System \\
    % IAS & Indicated Airspeed \\
    % ICAO & International Civil Aviation Organization \\
    % IFR & Instrument Flight Rules \\
    % ILS & Instrument Landing System \\
    % IPPD & Integrated Product and Process Development \\
    % IR & Infrared \\
    % IOS & International Standards Organization \\
    % JATO & Jet-Assisted Take off \\
    % KISS & KEEP IT SIMPLE, STUPID! (Ed Heinemmann) \\
    % LCC & Life-cycle cost \\
    % LOX & Liquid Oxygen \\
    % L\&P & Leakage and Protuberances \\
    % MDO & Multidisciplinary Design Optimization \\
    % MSL & Mean Sea Level \\
    % MTOW & Maximum Take-off Weight \\
    % MZFW & Maximum Zero Fuel Weight \\
    % NASA & National Aeronautics and Space Administration \\
    % NAV & Navigation \\
    % NS & Navier-Stokes (high-end CFD) \\
    % OEM & Original Equipment Manufacturer \\
    % OEI & One Engine Inoperative \\
    % OML & Outer Mold Lines \\
    % RCS & Radar Cross Section \\
    % RCS & Reaction Control System \\
    % RFI & Request for Information \\
    % RFP & Request for Proposals \\
    % RFQ & Request for Quotations \\
    % RPM & Revolutions per Minute \\
    % S\&C & Stability and Control \\
    % SL & Sea Level \\
    % SOP & Standard Operating Procedure \\
    % STOL & Short Take-off and Landing \\
    % TAS & True Airspeed \\
    % TVC & Thrust Vector Control \\
    % UAV & Unmanned Aerial Vehicle \\
    % VFR & Visual Flight Rules \\
    % VTOL & Vertical Take-off and Landing \\
    % ZFW & Zero Fuel Weight\\
\end{longtable*}}

\newpage
\section{List of Figures}


\newpage
\section{List of Tables}


\newpage
\section{Executive Summary}
\textcolor{red}{
\begin{itemize}
    \item Briefly discuss motivation for the aircraft, key requirements, and your design drivers.
    \item Briefly  summarize the proposed aircraft, including key  characteristics (e.g. general configuration, TOGW, primary dimensions), key performance metrics, key differentiators, and a cost summary.
    \item Include any requirements not met, struggling with, or have chosen to exceed.
    \item Discuss the near term milestones and provide a project timeline for the semester.
    \item Maximum length: one page, single-spaced if necessary.
\end{itemize}}

\pagenumbering{arabic}
\section{Introduction}
\textcolor{red}{
\begin{itemize}
    \item Discuss motivation, design objectives (e.g design drivers, key requirements), and a summary of key design characteristics and capabilities.
    \item This section should identify your niche (i.e. who your target customers are), convince the reader to read the rest of the report, and provide context for the remaining discussion.
    \item Discuss any unique attributes to your design philosophy.
    \item Start pagination as: 1, 2, 3,...
\end{itemize}}

\section{Concept of Operations}
\textcolor{red}{
\begin{itemize}
    \item Discuss requirements (including those from the RFP and any additional derived requirements), constraints, and the design mission.
    \begin{itemize}
        \item Include a table of requirements.
        \item Include a figure of the design mission with key mission segments labeled and details noted (e.g. cruise range and altitude).
    \end{itemize}
\end{itemize}}

\section{Sizing Analysis}
\textcolor{red}{\begin{itemize}
    \item Discuss the team’s similarity analysis and the rationale behind the aircraft selection.
    \item Discuss what data was extracted from the similarity analysis.
    \item Discuss a part by part build up and how it compares to the seed analysis
    \item Discuss initial sizing and constraint analysis.
    \begin{itemize}
        \item Include a table of key parameters used with justification for their values.
        \item Include discussion of modifications made to account for the hybrid-electric nature of your aircraft.
        \item Include at least takeoff, landing, cruise, service ceiling, and loiter constraints.
    \end{itemize}
    \item Include at least two trade studies that uses quantitative analysis to support design decisions made.
    \item Include the aircraft allocated and derived requirements and their justifications:
    \begin{itemize}
        \item $\frac{L}{D}$
        \item $C_{L,max}$
        \item SFC
        \item Weights
    \end{itemize}
\end{itemize}}

\section{Configuration}
\textcolor{red}{
\begin{itemize}
    \item Discuss external configuration alternatives considered and design choices made.
    \item Discuss selected aircraft configuration design (i.e. distinctions between variants, major features, design characteristics/objectives).
    \item Include at least one trade study that uses quantitative or qualitative analysis to support design decisions made.
    \item Discuss the landing gear philosophy.
    \item 3-View drawing (VSP acceptable).
    \item Discuss future work
\end{itemize}}

\section{Propulsion}
\textcolor{red}{
\begin{itemize}
    \item Discuss overall propulsion system philosophy/design/selection.
    \item Discuss future work.
\end{itemize}}

\section{Aerodynamics}
\textcolor{red}{
\begin{itemize}
    \item Discuss wing design, including reasoning.
    \item Discuss high-lift system, including reasoning.
    \item Discuss drag buildup (tabulated) used to support sizing analysis.
    \item Discuss future work.
\end{itemize}}

\section{Performance}
\textcolor{red}{
\begin{itemize}
    \item Discuss performance analysis for takeoff and cruise supporting the sizing analysis.
    \item Discuss future work.
\end{itemize}}

\section{Stability and Control}
\textcolor{red}{
\begin{itemize}
    \item Discuss any analysis supporting the sizing analysis.
    \item Discuss future work.
\end{itemize}}

\section{Structures and Loads}
\textcolor{red}{
\begin{itemize}
    \item Discuss any analysis supporting the sizing analysis.
    \item Discuss future work.
\end{itemize}}

\section{Mass Properties}
\textcolor{red}{
\begin{itemize}
    \item Dicuss any analysis supporting the sizing analysis.
    \begin{itemize}
        \item Mass property methods chosen for different parts of the aircraft.
        \item Estimate CG location.
    \end{itemize}
    \item Discuss future work.
\end{itemize}}

\section{Landing Gear}
\textcolor{red}{
\begin{itemize}
    \item Discuss landing gear placement and design approach.
    \item Discuss future work.
\end{itemize}}

\section{Systems}
\textcolor{red}{
\begin{itemize}
    \item Discuss any analysis supporting the sizing analysis.
    \item Discuss future work.
\end{itemize}}

\section{Cost Analysis}
\textcolor{red}{
\begin{itemize}
    \item Discuss any analysis supporting the sizing analysis.
    \item Discuss future work.
\end{itemize}}

\section{Conclusion}
\textcolor{red}{
\begin{itemize}
    \item Summarize your motivation, design objectives, proposed design, key performance metrics, and key differentiators.
    \item Discuss problems encountered (e.g. requirements not met) and recommendations for future study (if any).
    \item Summarize future work.
\end{itemize}}

\section{References}


\section{Appendix}


\end{document}
